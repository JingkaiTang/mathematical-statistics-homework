\section{数理统计的基本概念}

  \textbf{1.1} 设总体$X \sim N(\mu, \sigma^2)$,$X_{1}$,...,$X_{n}$为总体$X$的一个样本,$\bar{X}$为样本均值,如要
  $$ P\{|\bar{X} - \mu| < 1\} = 0.95 $$
  则样本容量$n$应取多大?
  
  \textbf{解:}
  由\textbf{定理 1.2.4}得$\bar{X} \sim N(\mu, \frac{\sigma^2}{n})$,所以$\frac{\sqrt{n} \cdot (\bar{X} - \mu)}{\sigma} \sim N(0, 1)$,又由条件$P\{|\bar{X} - \mu| < 1\} = 0.95$可得,$P\{|\frac{\sqrt{n} \cdot (\bar{X} - \mu)}{\sigma}| < \frac{\sqrt{n}}{\sigma}\} = 0.95$,有
  \[ \begin{split}
    \alpha &= (1 - 0.95) \div 2\\
    &= 0.025\\
    \mu{}_{\alpha{}=0.025} &= 1.96\\
    1.96 &< \frac{\sqrt{n}}{\sigma}\\
    n &= \lceil (1.96 \cdot \sigma)^2 \rceil
  \end{split} \]
