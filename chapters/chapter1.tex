\section{数理统计的基本概念}

  \xtl{1.1} 设总体$X \sim N(\mu, \sigma^2)$,$X_{1}$,...,$X_{n}$为总体$X$的一个样本,$\bar{X}$为样本均值,如要
  $$ P\{|\bar{X} - \mu| < 1\} = 0.95 $$
  则样本容量$n$应取多大?
  
  \xsv
  由\xtm{1.2.4}得$\bar{X} \sim N(\mu, \frac{\sigma^2}{n})$,所以$\frac{\sqrt{n} \cdot (\bar{X} - \mu)}{\sigma} \sim N(0, 1)$,又由条件$P\{|\bar{X} - \mu| < 1\} = 0.95$可得,$P\{|\frac{\sqrt{n} \cdot (\bar{X} - \mu)}{\sigma}| < \frac{\sqrt{n}}{\sigma}\} = 0.95$,有
  \[ \begin{split}
    \alpha &= (1 - 0.95) \div 2\\
    &= 0.025\\
    \mu{}_{\alpha{}=0.025} &= 1.96\\
    1.96 &< \frac{\sqrt{n}}{\sigma}\\
    n &= \lceil (1.96 \cdot \sigma)^2 \rceil
  \end{split} \]

  \xtl{1.2} 设电子元件的寿命(小时)$X \sim Exp(0.0015)$,独立测试6个元件并记下它们的失效时间,试求:
  
  (1) 至800小时,没有一个元件失效的概率;
  
  (2) 至3000小时,所有元件都失效的概率.
  
  \xsv
  (1) 由题意知$F(x) = e^{-0.0015 \cdot x}$,则
  \[ \begin{split}
    P\{X_{1}, X_{2}, X_{3}, X_{4}, X_{5}, X_{6}; X \geq 800\} &= (F(800))^6\\
    &= (e^{-0.0015 \times 800})^6\\
    &= e^{-7.2}
  \end{split} \]
  
  (2)
  \[ \begin{split}
    P\{X_{1}, X_{2}, X_{3}, X_{4}, X_{5}, X_{6}; X < 3000\} &= (1 - F(3000))^6\\
    &= (1 - e^{-0.0015 \times 3000})^6\\
    &= (1 - e^{-4.5})^6
  \end{split} \]
   
  \xtl{1.4} 设总体$X$服从对数正态分布,即$X$具有概率密度
  \[ 
    f(x) = 
    \begin{cases}
      \frac{1}{x \sqrt{2\pi} \sigma} e^{-\frac{(\ln{x} - \mu)^2}{2 \sigma^2}},\quad &0 < x < \infty\\
      0,\quad &\text{其他}
    \end{cases}
  \]
  $X_1, ..., X_n$为总体$X$的一个样本,试写出$X_1, ...,X_n$的联合概率密度.
  
  \xsv
  \def\teq14{
    \frac{
      e^{-\frac{1}{2 \sigma^2} \cdot \sumin{(\ln{x_i} - \mu)^2}}
    }{
      (2 \pi \sigma^2)^{\frac{n}{2}} \cdot \prodin{x_i}
    }
  }
  \[ \begin{split}
    x > 0,\quad
    \prodin{f(x_i)} &= \prodin{\frac{1}{x_i \sqrt{2\pi} \sigma} \cdot e^{-\frac{(\ln{x_i} - \mu)^2}{2 \sigma^2}}}\\
    &= \teq14
  \end{split} \]
  则:
  \[
    f(x_1, ..., x_n) =
    \begin{cases}
      \teq14,\quad &0 < x < \infty\\
      0,\quad &\text{其他}
    \end{cases}
  \]
  
  \xtl{1.6} 证明下列等式
  
  (1) $\sumin{(X_i - \mu)^2} = \sumin{(X_i - \bar{X})^2 + n(\bar{X} - \mu)^2}$;
  
  (2) $\sumin{(X_i - \bar{X})^2} = \sum\limits_{i=1}^{n}{X_i^2 - n\bar{X}^2}$.

  \xsv
  (1)
  \[ \begin{split}
    \sumin{(X_i - \mu)^2} &= \sumin{[(X_i - \bar{X}) + (\bar{X} - \mu)]^2}\\
    &= \sumin{[(X_i - \bar{X})^2 + 2(X_i - \bar{X})(\bar{X} - \mu) + (\bar{X} - \mu)^2]}\\
    &= \sumin{(X_i - \bar{X})^2} + 2(\bar{X} - \mu) \cdot \sumin{(X_i - \bar{X}} + \sumin{(\bar{X} - \mu)^2}\\
    &= \sumin{(X_i - \bar{X})^2} + n(\bar{X} - \mu)^2
  \end{split} \]
  
  (2)
  \[ \begin{split}
    \sumin{(X_i - \bar{X})^2} &= \sumin{(X_i - \bar{X})(X_i - \bar{X})}\\
    &= \sumin{[(X_i - \bar{X})X_i - (X_i - \bar{X})\bar{X}]}\\
    &= \sumin{[X_i^2 - \bar{X} \cdot X_i - (X_i - \bar{X})\bar{X}]}\\
    &= \sumin{X_i^2} - \bar{X}\sumin{X_i} - \bar{X}\sumin{(X_i - \bar{X})}\\
    &= \sumin{X_i^2} - n\bar{X}^2
  \end{split} \]


  


